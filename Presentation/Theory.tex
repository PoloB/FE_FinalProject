\section{Theory}

\subsection{Barrier option}
\begin{frame}
\frametitle{\insertsection / \small{\insertsubsection}}
\begin{itemize}
	\item Knock-in :
	\begin{itemize}
		%\item can be activated only once
		\item observed every end of business days (so we always generate closing prices)
		\item active if at least one of the indexes drop under 60\%
		\item activation means the repayment will be the min of both index price percentages and 100\%
		\item no more protection for the client if the knock-in activates
	\end{itemize}
	\item Knock-out :
	\begin{itemize}
		%\item can be activated only once
		\item observed every valuation day (for repayments) 
		\item active if only one index raise over 105\%
		\item activation means the repayment is done before the maturity at 100\% rate (contract stopped)
		\item protection for the seller because there is high probability they will repay 100\% at maturity, so they save some interest payments (they can raise the interest payments to lure clients)
	\end{itemize}
\end{itemize}

\subsection{Repayment cases}
\begin{frame}
\frametitle{\insertsection / \insertsubsection}
\begin{itemize}
	\item If at a 
	\item If none of the market price ends a day (closing price) at 60\% of its initial value (called base or standard) : the knock-in is not active and the reimbursement is total.
\end{itemize}

\subsection{Risks}
\begin{frame}
\frametitle{\insertsection / \insertsubsection}
\begin{itemize}
	\item The product is rated 
\end{itemize}

\centering
%\includegraphics[width=0.65\textwidth]{BinomialTree}
\end{frame}

\subsection{Estimation of parameters}
\begin{frame}
\frametitle{\insertsection / \insertsubsection}
\begin{itemize}
	\item We estimate over 3 or 6 months the mean $\mu$, the volatility $\sigma$ of each market price, plus the correlation $\rho$ between them both.
	\item The formulas are, with $u_{j,i} = \log_e\bigl(\frac{S_{j,i-1}}{S_{j,i}}\bigr)$
\end{itemize}
\begin{align*}
	\hat{\mu}_j    &= \frac{1}{m} \sum_{i=1}^m u_{j,i} \\
	\hat{\sigma}_j &= \sqrt{\frac{1}{m-1} \sum_{i=1}^m (u_{j,i}-\hat{\mu_j})^2} \\
	\hat{\rho}   \;&= \frac{\dfrac{1}{m}\sum_{i=1}^m u_{1,i}u_{2,i}}{\hat{\sigma}_1 \hat{\sigma}_2}
\end{align*}
\end{frame}

\subsection{Geometric Brownian Motion}
\begin{frame}
\frametitle{\insertsection / \insertsubsection}
\begin{itemize}
  \item Notations :
	\begin{itemize}
		\item S is the closing price of the index price% (either S\&P500 or Nikkei225)
		\item $\Delta t$ is the constant time step%, expressed in years ($\approx \frac{1}{252}$) 
		\item $\mu$ is the estimated mean of the historical data of the index price
		\item $\sigma$ is the estimated volatility of the same historical data
		\item $\xi$ is the independent standard normal distribution
	\end{itemize}
	\item GBM in its exact discretization form:
\end{itemize}
\begin{gather*}
	S_j(t+\Delta t) = S_j(t) \;
		e^{\bigl(\mu_j-\frac{{\sigma_j}^2}{2}\bigr)\Delta t+\sigma_j \sqrt{\Delta t}\mathcal{E}_j} 
	\\
	\\
	\mbox{with }
	\left \{
	\begin{array}{ll}
    \mathcal{E}_1 & = \xi_1 \\
    \mathcal{E}_2 & = \rho\xi_1 + \sqrt{1-\rho^2}\xi_2
  \end{array}
	\right.
\end{gather*}

%\begin{itemize}
	%\item []
%\end{itemize}
\end{frame}

