\section{Theory}


\subsection{Barrier option}
\begin{frame}
\myframetitle{}
\begin{itemize}
	\item Knock-in :
	\begin{itemize}
		%\item can be activated only once
		\item observed every end of business days (so we always generate closing prices)
		\item active if at least one of the indexes drops under 60\%
		\item activation means the repayment will be the min of both index price percentages and 100\%
		\item no more protection for the client if the knock-in activates
	\end{itemize}
	\item []
	\item Knock-out :
	\begin{itemize}
		%\item can be activated only once
		\item observed every valuation day of interest payments
		\item active if both indexes raise over 105\%
		\item activation means the repayment is done before the maturity at 100\%-rate (contract stopped)
		\item protection for the seller because there is high probability they will repay 100\% at maturity, so they save some interest payments (they can raise the interest payments to lure clients)
	\end{itemize}
\end{itemize}
\end{frame}

\subsection{Repayment cases}
\begin{frame}
\myframetitle{}
\begin{itemize}
		\item []
		\item []
\end{itemize}
\begin{table}[c]
	\centering
	\begin{tabular}{|c|c|c|c|} 
    \hline
    \multicolumn{2}{|c|}{Repayment rate} & \multicolumn{2}{c|}{Knock-in} 										 \\
		\cline{3-4}
    \multicolumn{2}{|c|}{}  						 & Not activated 	& Activated 											 \\
    \hline
		Knock-out & Not activated 					 & 1 							& $min(\frac{S_{j,T}}{S_{j,0}},1)$ \\
		\cline{2-4}
							& Activated 							 & 1 							&	1 															 \\
    \hline
	\end{tabular}
	\caption{The different rates of repayment cases, depending on realization of knock-in and knock-out}
\end{table}

\end{frame}

\subsection{Risks}
\begin{frame}[c]
\myframetitle{}
\begin{itemize}
	\item The product has a credit rating of A (S\&P), so it has an average cumulative default rate of 0.34 over 3 years
	\item []
	\item The knock-out can stop prior to maturity (need to reinvest)
	\item []
	\item Worst case scenario: at least one of both indexes is very low in 3 years
\end{itemize}
\end{frame}

\subsection{Geometric Brownian Motion}
\begin{frame}
\myframetitle{}
\begin{itemize}
  \item Notations:
	\begin{itemize}
		\item S is the closing price of an index% (either S\&P500 or Nikkei225)
		\item $\Delta t$ is the constant time step%, expressed in years ($\approx \frac{1}{252}$) 
		\item $\mu$ is the estimated mean of the historical data of an index price
		\item $\sigma$ is the estimated volatility of the same historical data
		\item $\xi$ is the independent standard normal distribution
		\item $\rho$ is the correlation between both indexes
	\end{itemize}
	\item []
	\item GBM in its exact discretization form:
\end{itemize}
\begin{gather*}
	S_j(t+\Delta t) = S_j(t) \;
		e^{\bigl(\mu_j-\frac{{\sigma_j}^2}{2}\bigr)\Delta t+\sigma_j \sqrt{\Delta t}\mathcal{E}_j} 
	\\
	\\
	\mbox{with }
	\left \{
	\begin{array}{ll}
    \mathcal{E}_1 & = \xi_1 \\
    \mathcal{E}_2 & = \rho\xi_1 + \sqrt{1-\rho^2}\xi_2
  \end{array}
	\right.
\end{gather*}
\end{frame}

\subsection{Estimation of parameters}
\begin{frame}
\myframetitle{}
\begin{itemize}
	\item We estimate over 3 months the mean $\mu$, the volatility $\sigma$ of each market price, plus the correlation $\rho$ between them both.
	\item The formulas are, with the log-returns over one day $u_{j,i} = \log_e\bigl(\frac{S_{j,i-1}}{S_{j,i}}\bigr)$
\end{itemize}
\begin{align*}
	\hat{\mu}_j    &= \frac{1}{m} \sum_{i=1}^m u_{j,i} \\
	\hat{\sigma}_j &= \sqrt{\frac{1}{m-1} \sum_{i=1}^m (u_{j,i}-\hat{\mu_j})^2} \\
	\hat{\rho}   \;&= \frac{\dfrac{1}{m}\sum_{i=1}^m u_{1,i}u_{2,i}}{\hat{\sigma}_1 \hat{\sigma}_2}
\end{align*}
\end{frame}
