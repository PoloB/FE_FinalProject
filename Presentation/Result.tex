\section{Simple case}

\subsection{Numerical efficiency}
\begin{frame}
\frametitle{\insertsection / \insertsubsection}
\begin{figure}
	\centering
	%\includegraphics[width=0.9\textwidth]{}
\end{figure}
\end{frame}

\subsection{Probability of loss}
\begin{frame}
\frametitle{\insertsection / \insertsubsection}
\begin{itemize}
	\item []
\end{itemize}
\end{frame}

\subsection{Histogram of the return}
\begin{frame}
\frametitle{\insertsection / \insertsubsection}
\begin{itemize}
	\item []
\end{itemize}
\end{frame}

\subsection{Sensitivity analysis}
\begin{frame}
\frametitle{\insertsection / \insertsubsection}
\begin{itemize}
	\item []
\end{itemize}
\end{frame}

\subsection{Comparison with Product A}
\begin{frame}
\frametitle{\insertsection / \insertsubsection}
\begin{itemize}
	\item 2 indexes : EuroSTOXX50 \& Nikkei225, T = 5 years, Knock-out 49\% 
	$$
	r = 
	\left \{
	\begin{array}{ll}
    1\%   & \mbox{if one of the index is under 80\% of its initial value} \\
		5.7\% & \mbox{else, or it is the first interest payment}
  \end{array}
	\right.
	$$
\end{itemize}
\end{frame}